Know-how and expressivism are usually regarded as disjoint topics, belonging to distant areas of philosophy. This paper argues that, despite obvious differences, the two debates have important similarities. In particular, semantic and conceptual tools developed by expressivists can be exported to the know-how debate. On the one hand, some of the expressivists' semantic resources can be used to deflect Stanley and Williamson's influential argument for factualism about know-how: the claim that knowing how to do something consists in knowing a fact. On the other, expressivism provides the resources to formulate a nonfactualist account of know-how. On this account, know-how has a kind of nonpropositional content and plays the role of guiding performance of action, rather than recording information from the environment.
\endinput