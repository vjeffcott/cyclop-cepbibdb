Two canonical negatives in Mandarin, \textit{mei} and \textit{bu}, display an asymmetry with respect to the presence of neg-raising inferences. In particular, \textit{mei} prefers non-neg-raising readings, while \textit{bu}, unless attaching to a functional category, is forced to be interpreted as neg-raising. This paper aims to explore an approach to address this asymmetry based on interactions between negation and aspect in both syntax and semantics. I argue that the asymmetry between \textit{mei} and \textit{bu} is resulted from their syntactic positions relative to aspect, and their licensing conditions especially selections of event variable binders.
\endinput