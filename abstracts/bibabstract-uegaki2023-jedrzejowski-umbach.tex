The predicate \textit{doubt} exhibits a distinctive selectional restriction: it is compatible with declarative and \textit{whether}-complements but is incompatible with constituent \textit{wh}-complements. The interpretation of a \textit{whether}-complement under \textit{doubt} is also puzzling, as \textit{doubt whether p} is almost—but not completely—synonymous with \textit{doubt that p}. This chapter presents a semantic account of these facts, based on the mechanisms of highlighting and exhaustification. \textit{Doubt} expresses an existential doxastic attitude toward the negation of the highlighted content of the complement while having presuppositions that are sensitive to the ordinary content of the complement. Since \textit{that p} and \textit{whether p} are equivalent in the highlighted content but non-equivalent in the ordinary content, the semantics explains fine-grained differences in interpretations between \textit{doubt whether p} and \textit{doubt that p}. Furthermore, due to the absence of a stronger scalemate, the interpretation of \textit{doubt that}/\textit{whether p} undergoes strengthening via exhaustification, akin to the behavior of scaleless modals.
\endinput