This volume presents fourteen case studies of \textit{wh}-clauses which are non-interrogative—lacking an interrogative meaning—and (mostly) subordinate. Moreover, the major part of the studies focuses on cases in which the meaning of the \textit{wh}-word seems to deviate from the literal meaning of the \textit{wh}-word (referring to persons, things, places, times, etc.). These include subordinate manner \textit{wh}-clauses that have a declarative-like meaning as well as locative \textit{wh}-clauses expressing kinds and headed relatives that serve as recognitional cues, to name just a few. While regular interrogative embedding has been widely studied in recent years, little is known about the circumstances under which non-interrogative (subordinate) \textit{wh}-clauses are licensed and the reason why some \textit{wh}-phrases can be polyfunctional whereas others are restricted to their literal meaning. The chapters in the book provide novel insights into the various ways in which interrogative \textit{wh}-phrases can be used in introducing complements, relative clauses, and adverbial clauses. The chapters combine cross-linguistic variations of subordination patterns with formal semantic and syntactic analyses, showing how the meanings associated with the \textit{wh}-words are exploited beyond their standard distribution. In the chapters, a wide range of languages are considered including Basque, Czech, Dutch, English, French, German, Mandarin, Romanian, Russian, Taiwan Southern Min, and Vietnamese. The findings have general implications both for the phenomenon of subordination and for the relationship between form and meaning in the field of \textit{wh}-clauses.
\endinput