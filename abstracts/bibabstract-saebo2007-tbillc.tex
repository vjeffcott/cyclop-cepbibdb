It is a well-known fact that only factive propositional attitude predicates are felicitous with \textit{wh}- (indirect question) complements. It has also been noted that so-called emotive factive predicates are only felicitous with some, not all, indirect question complements. But the reasons for these two constraints have remained unclear. I propose a competition-based explanation in terms of optimality theoretic pragmatics: Due to the competition with factive predicates, predicates like \textit{believe} are infelicitous with complements automatically verifying the factive presupposition; and emotive factive predicates are infelicitous with \textit{wh}- complements to the extent that these complements compete with more informative \textit{that} complements. To arrive at these results, it is necessary to assume an analysis of questions on which they denote propositions and to be more careful than has been customary about the formulation of the presuppositions of factive and what I call super-factive predicates.
\endinput