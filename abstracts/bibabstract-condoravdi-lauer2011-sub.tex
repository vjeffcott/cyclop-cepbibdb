Searle (\cite*{searle1989-lingphil}) posits a set of adequacy criteria for any account of the meaning and use of performative verbs, such as order or promise. Central among them are: (a) performative utterances are performances of the act named by the performative verb; (b) performative utterances are self-verifying; (c) performative utterances achieve (a) and (b) in virtue of their literal meaning. He then argues that the fundamental problem with assertoric accounts of performatives is that they fail (b), and hence (a), because being committed to having an intention does not guarantee having that intention. Relying on a uniform meaning for verbs on their reportative and performative uses, we propose an assertoric analysis of performative utterances that does not require an actual intention for deriving (b), and hence can meet (a) and (c).
\endinput