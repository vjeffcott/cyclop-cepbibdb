This dissertation focuses on what it means to \textit{think that} or to \textit{think whether something is the case}.

First, I show that the type of clause that \textit{think} combines with makes a difference in the kind of attitude a thought report ascribes, and in the kind of eventuality that it describes. With a declarative complement, \textit{think} entails belief and introduces an eventuality description that may (but need not be) stative. With a question complement, \textit{think} introduces an eventuality description that is necessarily dynamic, and often deliberative. In this case, there is no entailment of belief, but one of agnosticism and curiosity about the answer to the embedded question. This, and a second generalization that governs the attitude related and the aspectual properties of \textit{think that} and of \textit{think whether} hold cross-linguistically.

Second, I attempt to square the observation that \textit{think whether} is grammatical and necessarily dynamic with two recent proposals that predict \textit{that think} should be ungrammatical with embedded questions (Mayr, \cite*{mayr2019-nls}; Theiler et al., \cite*{theiler-roelofsen-aloni2019-nls}). As these go, the excluded middle presupposition, which is associated with \textit{think} to derive neg(ative)-raising with declaratives, gives rise to an anomalous meaning with embedded questions, which is perceived as ungrammaticality. I argue that question embedding and neg-raising do exclude each other, pace White (accepted), but that this is because the former requires \textit{think} to introduce a dynamic description, and the latter requires a stative one (Xiang, \cite*{xiang2013-plc}; Bervoets, \cite*{bervoets2014-thesis}, \cite*{bervoets2020}). I stress that \textit{think whether} is often unacceptable in the sentence frames in which we have been trying to observe it—for example, with the verb in the present simple—but argue that such restrictions are reducible for the most part to interpretive restrictions on dynamic predicates in those frames.

Third, we need to derive the attitude related and the aspectual alternations that \textit{think} gives rise to with declarative and question complements. I propose that \textit{think}'s attitude component encodes the entertain modality from inquisitive semantics (Ciardelli and Roelofsen, \cite*{ciardelli-roelofsen2015-synthese}, a.o.), which is equivalent to belief in the declarative case and compatible with an inquisitive attitude in the quesvi tion case. Turning to the aspectual alternation, I propose to structure attitude eventualities with embedded clause denotations by relating subeventualities of the former and the alternatives provided by the latter. Declaratives provide a single alternative, which makes it possible to construct a divisive (hence stative) predicate. Questions, on the other hand, provide multiple alternatives, which forces \textit{think whether} to introduce a non-divisive (hence non-stative) predicate.

Finally, I sketch out some ways in which attitude predicates vary in terms of their attitude related and lexical aspectual properties. I ask whether \textit{believe that} and \textit{believe wh-} could be given a treatment similar to \textit{think}. I point out that there are predicates like \textit{know}, \textit{remember} and \textit{agree} that may or must remain stative in question embedding, and speculate whether presupposing truth or belief might be enabling this option. I end with a typology of predicates that should all be possible if the stative/dynamic alternation seen for \textit{think} were free, observe that many are missing and characterize the ones that are.
\endinput