Control, typically defined as a specific referential dependency between the null-subject of a non-finite embedded clause and a co-dependent of the matrix predicate, has been subject to extensive research in the last 50 years. While there is a broad consensus that a distinction between Obligatory Control (OC), Non-Obligatory Control (NOC) and No Control (NC) is useful and necessary to cover the range of relevant empirical phenomena, there is still less agreement regarding their proper analyses. In light of this ongoing discussion, the articles collected in this volume provide a cross-linguistic perspective on central questions in the study of control, with a focus on non-canonical control phenomena. This includes cases which show NOC or NC in complement clauses or OC in adjunct clauses, cases in which the controlled subject is not in an infinitival clause, or in which there is no unique controller in OC (i.e.\ partial control, split control, or other types of controllers). Based on empirical generalizations from a wide range of languages, this volume provides insights into cross-linguistic variation in the interplay of different components of control such as the properties of the constituent hosting the controlled subject, the syntactic and lexical properties of the matrix predicate as well as restrictions on the controller, thereby furthering our empirical and theoretical understanding of control in grammar.
\endinput