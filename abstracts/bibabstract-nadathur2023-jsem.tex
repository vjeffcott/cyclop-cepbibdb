\textbf{Implicative verbs} (e.g., \textit{manage}, \textit{dare}) are characterized by complement inferences (Karttunen, \cite*{karttunen1971-language}). English \textit{manage} entails its complement; the entailment reverses polarity with matrix negation, and is accompanied by a projective inference to the complement’s non-triviality (Coleman, \cite*{coleman1975-bls}; Karttunen \& Peters, \cite*{karttunen-peters1979-dinneen-oh}). I use data from Finnish and English to argue that the implicative inferential profile is derived from backgrounded relations of \textit{causal necessity} and \textit{causal sufficiency} (defined over the structure of a formal causal model; Pearl \cite*{pearl2000}; Schulz \cite*{schulz2010-synthese}) which link the lexical content of an implicative verb to the realization of its complement. The proposal builds on Baglini \& Francez’s (\cite*{baglini-francez2015-jsem}) causal analysis of \textit{manage}, but significantly revises the earlier proposal to offer a treatment which accounts not only for English \textit{manage}, but extends to the lexical semantics of the full implicative class, including ‘polarity-reversing’ verbs like \textit{fail}, lexically specific verbs like \textit{dare}, and their Finnish counterparts. Unlike earlier analyses, the proposed causal semantics also provides a natural explanation of the commonalities between two-way entailing verbs like \textit{manage} and a related class of weaker ‘one-way’ implicatives such as Finnish \textit{jaksaa} (‘have the strength’), which entail complement truth values under only one matrix polarity, but generate strong pragmatic implicatures in the two-way implicative pattern under the non-entailing polarity.
\endinput