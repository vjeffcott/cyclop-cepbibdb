Karttunen observed that, if the complement of an attitude sentence presupposes $p$, then that sentence as a whole presupposes that the attitude–holder believes $p$. I attempt to derive some representative instances of this generalization from suitable assumptions about the lexical semantics of attitude predicates. The enterprise is carried out in a framework of context change semantics, which incorporates Stalnaker's suggestion that presupposition projection results from the stepwise fashion in which information is updated in response to complex utterances. The empirical focus is on predicates of desire and on the contribution of counterfactual mood.
\endinput