Since \textcite{saebo1985-report,saebo2001-fery-sternefeld} drew the attention of formal semanticists to the compositionality problems raised by anankastic conditionals like \textit{If you want to go to Harlem, you have to take the A train}, a number of authors have proposed analyses tailor-made for such conditionals. We demonstrate that the seemingly puzzling properties of anankastic conditionals in fact show up independently from each other within a wider range of conditionals, which we call ‘near-anankastic’. While they do not have the means-of implication typically associated with anankastics, near-anankastics give rise to their own special additional implications. As a crucial ingredient for a unified account, we provide a new analysis of the semantics of the desire predicate in the antecedent — an issue that has not been adequately pursued in the previous literature. We claim that \textit{want} has an independently motivated reading on which it predicates the existence of an \textit{action-relevant preference} \parencite{condoravdi-lauer2011-sub,condoravdi-lauer2012-eiss,lauer2013-thesis}. We then show that the semantically determined interpretation of anankastic and near-anankastic conditionals arises, predictably and compositionally, from a range of interacting factors that are at play in the interpretation of conditional sentences more generally. The special implications associated with each kind of conditional arise pragmatically. Anankastic and near-anankastic conditionals alike turn out to be just what they seem: regular, hypothetical, indicative conditionals.
\endinput