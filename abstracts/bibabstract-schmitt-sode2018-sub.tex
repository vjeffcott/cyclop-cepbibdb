German \textit{wissen} (‘know’) can embed both finite clauses (‘\textit{wissen}-\textsc{fin}’) as well as infinitives (‘\textit{wissen}-\textsc{inf}’). Based on novel empirical observations, we argue that \textit{wissen}-\textsc{inf} cannot be reduced to the standard analysis of \textit{wissen}-\textsc{fin}, i.e. that \textit{wissen} with infinitival complements does not involve a propositional attitude. As cross-linguistic evidence suggests that German \textit{wissen} is not ambiguous, it follows that \textit{wissen}-\textsc{fin} cannot denote a propositional attitude, either. Accordingly, we require a new, uniform meaning for \textit{wissen}. We derive this meaning by first considering \textit{wissen}-\textsc{inf}, arguing that it combines semantic properties of ability modals with semantic properties of implicative verbs and \textit{enough to}-constructions. We then show that these properties can also be used to characterize \textit{wissen}-\textsc{fin}, as long as certain nonstandard assumptions are made about the denotation of the complement. This gives us a new, unified analysis of \textit{wissen} and also helps to explain some properties of this verb (with both kinds of complements) that traditional analyses cannot account for.
\endinput