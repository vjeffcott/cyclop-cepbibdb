In addition to the so-called `factive' verbs, which presuppose the truth of their complement sentence, a number of other interesting classes of verbs take sentential complements. `Implicative' verbs, such as manage, also involve presuppositions, although in a different way. An asserted main sentence with one of these verbs as predicate commits the speaker to an implied proposition which consists of the complement sentence as augmented by the tense and other modifiers of the main sentence. Questioning a sentence with an implicative predicate amounts to questioning that implied proposition. According to the proposed analysis, an implicative main verb carries a presupposition of some necessary and sufficient condition which alone determines whether the event described in the complement took place. The main sentence can be looked upon as a statement about whether this decisive condition is fulfilled, and under what spatial and temporal circumstances. From an affirmative assertion, it can legitimately be inferred that the implied proposition is asserted to be true; from a negative assertion, that it is asserted to be false.
\endinput