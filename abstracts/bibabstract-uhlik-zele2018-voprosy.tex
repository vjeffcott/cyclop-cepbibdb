\begin{otherlanguage}{russian}
В статье предпринята попытка проанализировать синтаксические и семантические особенности слов \textit{želeti} ‘хотеть, желать’, \textit{hoteti} ‘хотеть, желать (с директивным значением)’ и \textit{zahtevati} ‘требовать’, типичных глаголов желания и побуждения в словенском языке. Рассмотрены различные типы реализации их дополнений, в первую очередь актантные придаточные, вводимые союзами \textit{da} и \textit{naj}. Затронут вопрос взаимозаменяемости придаточных и инфинитивных дополнений. Проанализированы семантические, синтаксические и прагматические критерии, влияющие на выбор наклонения в придаточном предложении. В анализе выделяются три типа конструкций с союзами \textit{da} и \textit{naj} в независимых и зависимых предложениях: 1) \textit{da} + индикатив, 2) \textit{da} + сослагательное наклонение, 3) \textit{naj} + индикатив. Употребление конструкций связано с разными значениями: в \textit{da}-предложениях сослагательное наклонение выражает оптативное значение, в то время как \textit{da} и \textit{naj} с индикативом передают побуждение.
\end{otherlanguage}

\begin{otherlanguage}{english}
The article treats the syntactic and semantic peculiarities of the three primary representatives of Slovenian desiderative and manipulative verbs, viz. \textit{želeti} ‘wish’, \textit{hoteti} ‘want’, and \textit{zahtevati} ‘demand’. The main focus is on the various types of complements that they take, specifically the sentential complements introduced by complementizers \textit{da} and \textit{naj}, which function as objects to matrix predicates. Additional attention is devoted to the interchangeability of sentential and infinitive complements. The article analyzes the semantic, syntactic, and pragmatic criteria that influence the choice of a particular sentential complement, as well as the parameters determining the use of the complementizer that relates the clause complement to the matrix predicate. The analysis focuses on three types of construction with sentence connectives da and naj in main and dependent clauses: 1) \textit{da} + the indicative mood, 2) \textit{da} + the conditional mood, 3) \textit{naj} + the indicative mood. The use of such sentence structures is connected with different meanings that the combinations achieve: in the so-called da-constructions, the conditional has an optative reading, while both \textit{da} and \textit{naj} function as hortatives if co-occurring with the indicative.
\end{otherlanguage}
\endinput