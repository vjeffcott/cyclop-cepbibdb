The paper presents and examines a previously undescribed puzzle concerning the syntactic distribution of Russian mandative verbs (\textit{velet’} ‘order’, \textit{razrešit’} ‘allow’) and non-verbal deontic modals: these predicates exhibit dual behavior as they embed non-finite clauses with either implicit obligatorily controlled (PRO) or overt referential (DP) subjects. The ambiguity holds for the same native speakers and no detectable difference in terms of the Tense – Agreement characteristics can be found between infinitival constituents with DP/PRO subjects. To account for this phenomenon, I propose, first, to analyze mandative verbs as lexical realizations of a verb of communication that embeds a silent deontic modal head; the latter, in turn, takes a clausal proposition as its complement. Second, I demonstrate that the reported DP/PRO alternation is described by the following generalization: An embedded overt referential subject is allowed only when there is no potential dative DP controller available within the higher clause. In the spirit of the traditional Case theory, I argue that an embedded lexical subject must be Case licensed, and, since non-finite clauses are Case deficient, licensing may only be done by a higher (matrix) functional head, namely Appl⁰, which normally introduces an obligation Holder; thus, matrix Holders and lexical embedded subjects end up competing to receive Case from the same functional head. Finally, I show that, as no true subject raising happens, Case assignment proceeds long-distance over a CP boundary.
\endinput