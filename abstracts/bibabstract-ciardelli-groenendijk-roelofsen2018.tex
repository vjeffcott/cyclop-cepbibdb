There is an age-old tradition in linguistics and philosophy to identify the meaning of a entence with its truth-conditions. This can be explained by the fact that linguistic and philosophical investigations are usually carried out in a logical framework that was originally designed to characterize valid reasoning. Indeed, in order to determine whether an argument is valid, it suffices to know the truth-conditions of the premises and conclusion. However, argumentation is neither the sole, nor the primary function of language. One task that language more widely and ordinarily fulfils is to enable the exchange of information between conversational participants. Inquisitive semantics is a new logical framework for the analysis of this fundamental usage of language. Information exchange can be seen as a process of raising and resolving issues. Inquisitive semantics provides a new formal notion of issues, which makes it possible to model various concepts that are crucial for the analysis of linguistic information exchange in a more refined and more principled way than has been possible in previous frameworks. This book provides a detailed exposition of inquisitive semantics, and demonstrates its benefits with a range of applications in the semantic analysis of questions, coordination, modals, conditionals, and intonation.
\endinput