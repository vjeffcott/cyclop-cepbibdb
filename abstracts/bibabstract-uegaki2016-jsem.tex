This article proposes that \textsc{responsive predicates}—predicates that embed both interrogatives and declaratives, such as \textit{know}—select for questions (modeled as sets of propositions) rather than propositions. Declarative complements denote singleton proposition-sets, meaning that declarative-embedding is a special (‘trivialized’) case of question-embedding. This analysis is in contrast to the more standard analysis of responsive predicates that treats them as proposition-taking items and reduces embedded questions to propositions (\cite{groenendijk-stokhof1984-thesis,lahiri.u2002}).\par The argument is based on the semantic contrast between responsive predicates and predicates that only embed declaratives (e.g., \textit{believe}), when they take DP complements headed by content nouns \parencite{vendler1972}, as in \textit{John \textbf{knows} the rumor that Mary left} vs. \textit{John \textbf{believes} the rumor that Mary left}. Under the common assumption that responsive predicates are proposition-taking, any plausible predictive account of the entailment from \textit{John \textbf{believes} the rumor that Mary left} to \textit{John \textbf{believes} that Mary left} would run into an incorrect prediction that \textit{John \textbf{knows} the rumor that Mary left} would also entail \textit{John \textbf{knows} that Mary left}. On the other hand, if responsive predicates are question-taking, the contrast can be captured given an inventory of type-shifters that map entities to propositions and questions. It is also argued that the proposed analysis enables a natural semantic account of the selectional restrictions of attitude predicates: \textit{believe}-type predicates select for propositions, \textit{know}-type predicates select for questions, and \textit{ask}/\textit{wonder}-type predicates select for \textit{non-singleton} questions.
\endinput