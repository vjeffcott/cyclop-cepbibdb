Since Karttunen's {[\cite*{karttunen1971-language}]} original observations about the two-way implicative verb manage, there has been no consensus as to the descriptive and analytical characterization of the implicational behavior of this verb. This short paper presents a reevaluation of the relevant descriptive generalizations, and proposes an analysis, couched within the causal modeling approach to causation developed by Schulz (\cite*{schulz2010-synthese}). We propose that manage sentences presuppose the familiarity of a ‘catalyst’—a causally necessary but insufficient condition for the truth of the propositional argument of manage, and assert that the catalyst actually caused this truth. Various more specific implications mentioned in the literature—trying, difficulty, unlikelihood—are shown to be either special cases of our proposed presupposition, or else cancelable pragmatic inferences. The analysis is argued to account for some new empirical observations as well.
\endinput