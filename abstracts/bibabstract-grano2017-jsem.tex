Unlike belief and desire reports, intention reports (e.g. \textit{John intends to leave soon}) are not well studied in formal semantics. This article aims to begin to fill this gap, focusing on empirical similarities and differences that intention reports bear in relation to other attitude reports and to other expressions that involve intentional action. These empirical properties are shown to follow from the view that an intention report \textit{a intends p} denotes true iff \textit{a} has a maximally ranked \textsc{action-relevant} or \textsc{effective preference} (in the sense of \cite{condoravdi-lauer2016-semprag}) that \textit{a} bears the \textsc{responsibility} relation (in the sense of \cite{farkas1988-lingphil}) to \textit{p}. Also discussed are intention reports that do not involve syntactic control (e.g. \textit{John intends for Bill to leave soon}), which in some previous literature have been argued to involve coercion. The proposed semantics for \textit{intend} enables a coercion-free analysis of such sentences, and issues surrounding the choice between a coercion-free and a coercion-based approach are discussed.
\endinput