The complementation pattern of certain question-embedding predicates, such as \textit{know} and \textit{agree}, presents a puzzle for the compositional semantics of clausal complementation, as the predicates seem to be able to combine with two distinct types of semantic objects: propositions and questions. The traditional approach to the semantics of these predicates, where embedded questions are \textit{reduced} to propositions, faces two problems. First, it cannot account for the observation that \textit{know-wh} sentences require the subject not to believe any false answer to the embedded question. Second, it makes a problematic prediction concerning the interpretation of Predicates of Relevance, such as \textit{care} and \textit{matter}. We review three alternative approaches to the semantics of question-embedding predicates, i.e., the \textsc{proposition-to-question} reduction, the uniform approach and the \textsc{ambiguity} approach, and argue that only the Proposition-to-Question reduction and the uniform approach can deal with the interpretation of the Predicates of Relevance. The paper concludes with a remark on how lexical denotations of question-embedding predicates are constrained in general.
\endinput