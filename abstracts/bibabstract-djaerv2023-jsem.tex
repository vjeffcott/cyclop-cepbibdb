In the Hintikkan tradition, attitude verbs are viewed as relations \textit{between} individuals and propositions. Previous work on \textit{know} and believe with \textsc{Content DP}s like \textit{the rumour} has tended to treat \textit{know CP} vs. \textit{know DP} as polysemy. In this paper, I show that polysemy runs into conceptual and empirical problems, and propose instead a new decompositional approach to \textit{know}-verbs, which avoids polysemy; linking both \textit{know DP} and \textit{know CP} to the same lexical root, which describes, broadly speaking, acquaintance. This analysis thus provides an explicit and compositional morpho-semantic link between \textit{know DP} and \textit{know CP} that accounts for the interpretation of DPs as objects of acquaintance, and further captures the idea that knowledge, and factivity more broadly, is tied to acquaintance with a situation, the res (\cite{kratzer2002-lingphil}, a.o.). Based on detailed examination of the morpho-syntax and interpretation of DP and CP complements of \textit{believe}, I further show that DPs can either combine with \textit{believe} in the same fashion as CPs, as a direct object (saturating a propositional argument slot, as in \cite{uegaki2016-jsem}), or as an indirect object, via a type of attitudinal applicative (proposed here). The former option is defined for Content DPs and the latter for agentive DPs, so-called \textsc{Source DP}s. Together, these proposals account for the observation that the interpretation of \textit{believe DP} sentences varies depending on the type of DP (\textit{believe the rumour} vs. \textit{believe the referee}), whereas for \textit{know}-verbs, both types of DPs are interpreted as objects of acquaintance. At the core of the current proposal is the idea that verbs like \textit{know} and \textit{believe} differ fundamentally at the level of argument structure and internal morpho-semantic composition, and thus combine with DPs via different routes; contrary to uniform approaches to \textit{know} and \textit{believe}. Whereas \textit{believe}-verbs describe relations to intensional content, and require external licensing mechanisms to combine with DPs, \textit{know}-verbs describe complex relations, fundamentally anchored in the attitude holder’s acquaintance with (abstract or concrete) individuals in the world, and thus make reference to individuals as part of their argument structure. The current proposal also builds on and adds to previous insights about connections between factivity, DP-complementation, and question-embedding.
\endinput