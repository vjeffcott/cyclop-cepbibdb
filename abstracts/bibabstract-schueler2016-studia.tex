In this paper I argue that the presuppositions associated with factive predicates are triggered not by the predicates themselves, but by their complement clauses. Evidence is brought to bear from different clause types in English, some of which, such as some gerunds and nominalizations, cause the presuppositions to disappear in certain uses of the implicit conditional, or simple subjunctive, construction. I survey the attempt to analyze these facts by assuming that factive predicates trigger presuppositions, that their complement clauses do, and that both do, and find that the thesis that solely the complement clauses are responsible for the presuppositions achieves the maximum explanatory power. This hypothesis has further implications for the nature of presuppositions in general.
\endinput