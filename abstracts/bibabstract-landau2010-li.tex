Although they participate in control relations, implicit arguments are standardly viewed as unprojected ϑ-roles, absent from the syntax. I challenge this view and argue that implicit arguments are syntactically represented. The argument rests on the observation that implicit arguments can exercise partial control, and the claim that partial control must be encoded in the syntax (given plausible assumptions on the limits of lexical relations). I further argue that the syntactic constitution of implicit arguments is more impoverished than that of \textit{pro}, explaining their differential visibility to various syntactic processes.
\endinput